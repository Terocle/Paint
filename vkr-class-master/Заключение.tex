\section*{ЗАКЛЮЧЕНИЕ}
\addcontentsline{toc}{section}{ЗАКЛЮЧЕНИЕ}

Основной целью работы было создание интуитивно понятного графического редактора, способного удовлетворять потребности пользователей в обработке изображений. В ходе реализации проекта были определены ключевые компоненты и их взаимодействие. Важным этапом работы было проектирование удобного пользовательского интерфейса, который бы обеспечивал взаимодействие пользователя с редактором.

Реализация графического редактора включала в себя создание инструментов для рисования, редактирования и обработки изображений. Процесс разработки включал в себя тестирование функционала, что позволило выявить и устранить потенциальные ошибки.

Основные результаты работы:

\begin{enumerate}
\item Проведен анализ предметной области.
\item Разработана модель данных системы. Определены требования к системе.
\item Разработана архитектура приложения. Разработан пользовательский интерфейс приложения.
\item Приложение реализовано и протестировано. Проведено системное тестирование.
\end{enumerate}

Все требования, объявленные в техническом задании, были полностью реализованы, все задачи, поставленные в начале разработки проекта, были также решены.
