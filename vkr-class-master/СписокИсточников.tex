\addcontentsline{toc}{section}{СПИСОК ИСПОЛЬЗОВАННЫХ ИСТОЧНИКОВ}

\begin{thebibliography}{9}

    \bibitem{python_pocket_guide_online}
    Джиджак, Э. Python. Эффективное программирование. – ДМК Пресс, 2020. – 720 с. – ISBN 978-5-97060-824-0. – Текст~: прямой.
    \bibitem{python_mastering}
    Ван Розум, Г. Python. К вершинам мастерства / Г. Ван Розум. – БХВ-Петербург, 2019. – 672 с. – ISBN 978-5-4461-1360-3. – Текст~: прямой.
    \bibitem{python_data_science_book}
    Бизли, Д. Python. Наука о данных / Д. Бизли. – ДМК Пресс, 2019. – 384 с. – ISBN 978-5-97060-603-1. – Текст~: прямой.
    \bibitem{python_complex_tasks}
    Доусон, М. Python для сложных задач. Основы программирования / М. Доусон. – ДМК Пресс, 2020. – 512 с. – ISBN 978-5-907114-33-4. – Текст~: прямой.
    \bibitem{fluent_python_course}
    ВандерПлас, Дж. Fluent Python. Курс современного программиста / Дж. ВандерПлас. – ДМК Пресс, 2021. – 656 с. – ISBN 978-5-907114-94-5. – Текст~: прямой.
    \bibitem{python_comprehensive_guide}
    Мэтсес, Э. Python. Подробное руководство / Э. Мэтсес. – ДМК Пресс, 2016. – 528 с. – ISBN 978-5-94074-943-7. – Текст~: прямой.
    \bibitem{python_useful}
    Любанович, Б. Python. Самое необходимое / Б. Любанович. – Питер, 2018. – 352 с. – ISBN 978-5-496-02659-2. – Текст~: прямой.
    \bibitem{python_data_science}
    Саммерфильд, М. Программирование на Python в науках о данных / М. Саммерфильд. – ДМК Пресс, 2021. – 576 с. – ISBN 978-5-97060-987-2. – Текст~: прямой.
    \bibitem{python_express_course}
    Бейдер, Д. Python. Экспресс-курс / Д. Бейдер. – Питер, 2020. – 256 с. – ISBN 978-5-4461-1280-4. – Текст~: прямой.
    \bibitem{python_data_analysis}
    Маккини, У. Python и анализ данных / У. Маккини. – ДМК Пресс, 2017. – 536 с. – ISBN 978-5-97060-348-1. – Текст~: прямой.
    
\end{thebibliography}
