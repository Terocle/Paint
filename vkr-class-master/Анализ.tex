\section{Анализ предметной области}
\subsection{Описание предметной области}

История графических редакторов берет свое начало во второй половине 20 века и прошла через ряд значимых этапов. Одним из ключевых моментов стало появление первых графических редакторов, среди которых выделяется Sketchpad, созданный в 1963 году Иваном Сазерлендом в MIT. Эта программа открыла новые горизонты в возможностях создания и редактирования изображений с использованием графического интерфейса.

В последующие десятилетия появились инновационные графические редакторы, в числе которых Adobe Photoshop, ставший отраслевым стандартом для обработки изображений. Программы также стали более доступными, что способствовало распространению графического дизайна и творчества.

С развитием технологий возникли новые возможности, такие как векторная графика и трехмерное моделирование. Инструменты, вроде Adobe Illustrator и Blender, предоставляют пользователям широкий спектр функций для творческой работы и дизайна.

Сегодня графические редакторы стали неотъемлемой частью повседневной жизни, используемой для создания и редактирования изображений, дизайна интерфейсов, анимации и других творческих задач. Их эволюция свидетельствует о постоянном стремлении к улучшению возможностей обработки графики и созданию интуитивных инструментов для пользователей.

В результате графические редакторы охватывают разнообразные области, включая дизайн, искусство, рекламу и веб-разработку. Их роль в современном мире выходит за пределы профессиональных студий, охватывая любителей и тех, кто желает воплотить свои творческие идеи в цифровой форме.
\subsection{Графические редакторы, их классификация}

\textbf{Растровый графический редактор} -- специализированная программа, предназначенная для создания и обработки растровых изображений, то есть графики, которая в память компьютера записывается как набор точек (пикселей), которые образуют строки и столбцы, например, точка задается своими координатами (Х, У).

Любой пиксель имеет фиксированное положение и цвет. Хранение каждого пикселя требует некоторого количества бит информации, которое зависит от количества цветов в изображении. Таким образом, качество растровых изображений зависит от их размера (числа пикселей по горизонтали и вертикали) и количества цветов, которые могут принимать пиксели.

Растровые графические редакторы позволяют пользователю рисовать и редактировать изображения на экране компьютера, а также сохранять их в различных растровых форматах, и больше подходят для обработки и ретуширования фотографий, создания фотореалистичных иллюстраций, коллажей, и создания рисунков от руки с помощью графического планшета.

\textbf{Векторный графический редактор} - это программа, предназначенная для создания изображений высокой точности, например, чертежи или схемы, создания разметки страниц, типографии, логотипов, мультипликация, сложные геометрические шаблоны, технических иллюстраций, создания диаграмм и составления блок-схем.

Главными основными инструментами в векторном редакторе являются кривые Безье, позволяющие рисовать кривые (ломаные, прямые и гладкие) по сегментам с точным размещением узловых (опорных) точек и контролем над формой каждого сегмента.

