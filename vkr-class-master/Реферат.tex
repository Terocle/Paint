\abstract{РЕФЕРАТ}

Объем работы равен \formbytotal{lastpage}{страниц}{е}{ам}{ам}. Работа содержит \formbytotal{figurecnt}{иллюстраци}{ю}{и}{й}, \formbytotal{tablecnt}{таблиц}{у}{ы}{}, \arabic{bibcount} библиографических источников. Количество приложений – 1. Фрагменты исходного кода представлены в приложении А.

Перечень ключевых слов: графический редактор, интерфейс пользователя, инструменты рисования, обработка изображений, форматы файлов, растровая графика, классы, компонент, модуль, сущность, метод, пользователь.

Объектом разработки является графический редактор, включая его основные компоненты, функциональность, пользовательский интерфейс, инструменты рисования и редактирования, а также возможности обработки изображений.

Целью курсовой работы является разработка легкого в использовании графического редактора, ориентированный на широкий круг пользователей, предоставляющий базовый, но эффективный набор инструментов для создания и редактирования изображений.

В процессе создания сайта были выделены основные сущности путем создания информационных блоков, использованы классы и методы модулей, обеспечивающие работу с сущностями предметной области, а также корректную работу приложения, разработаны инструменты для взаимодействия с рабочей зоной программы.

При разработке сайта использовался язык программирования Python.

\selectlanguage{english}
\abstract{ABSTRACT}
  
The volume of work is \formbytotal{lastpage}{page}{}{s}{s}. The work contains \formbytotal{figurecnt}{illustration}{}{s}{s}, \formbytotal{tablecnt}{table}{}{s}{s}, \arabic{bibcount} bibliographic sources. The number of applications is 1. Fragments of the source code are presented in annex A.

The list of keywords: graphic editor, user interface, drawing tools, image processing, file formats, raster graphics, classes, component, module, entity, method, user.

The object of development is a graphic editor, including its main components, functionality, user interface, drawing and editing tools, as well as image processing capabilities.

The purpose of the course work is to develop an easy-to-use graphical editor aimed at a wide range of users, providing a basic but effective set of tools for creating and editing images.

In the process of creating the site, the main entities were identified by creating information blocks, classes and methods of modules were used to ensure work with the entities of the subject area, as well as the correct operation of the application, tools for interacting with the work area of the program were developed.

The Python programming language was used in the development of the site.
\selectlanguage{russian}
