\section*{ВВЕДЕНИЕ}
\addcontentsline{toc}{section}{ВВЕДЕНИЕ}

Графические редакторы представляют собой мощные инструменты для создания, редактирования и обработки изображений, играя важную роль в современном мире визуального искусства и дизайна. Они предоставляют пользователю уникальные возможности воплощения творческих идей, преобразуя пиксели на экране в произведения искусства.

Эти программы обладают разнообразным функционалом, включая рисование, обрезку, ретушь, добавление эффектов, редактирование цветов и многое другое. В современных условиях графические редакторы предоставляют широкий спектр инструментов для создания и редактирования фотографий, иллюстраций, дизайна интерфейсов, баннеров и других графических элементов.

Adobe Photoshop, GIMP, CorelDRAW, и множество других программ являются стандартами в этой области, предоставляя профессионалам и любителям возможность воплощать свои творческие идеи. С развитием технологий, графические редакторы становятся все более доступными и интуитивно понятными, что позволяет широкому кругу пользователей экспериментировать с изобразительными возможностями.

Благодаря графическим редакторам творчество становится более доступным, а визуальное восприятие обогащается уникальными изображениями, которые олицетворяют индивидуальность и талант их создателей.

\emph{Цель настоящей работы} – создание собственного графического редактора, способного удовлетворить потребности пользователей в редактировании и обработке изображений. Для достижения поставленной цели необходимо решить \emph{следующие задачи:}
\begin{itemize}
\item провести анализ предметной области;
\item разработать концептуальную модель приложения;
\item спроектировать приложение;
\item реализовать приложение средствами языка программирования Python.
\end{itemize}

\emph{Структура и объем работы.} Отчет состоит из введения, 4 разделов основной части, заключения, списка использованных источников, 2 приложений. Текст выпускной квалификационной работы равен \formbytotal{page}{страниц}{е}{ам}{ам}.

\emph{Во введении} сформулирована цель работы, поставлены задачи разработки, описана структура работы, приведено краткое содержание каждого из разделов.

\emph{В первом разделе} на стадии описания технической характеристики приводятся сведения о предметной области выбранной темы, историю развития программного обеспечения в данной теме.

\emph{Во втором разделе} на стадии технического задания приводятся требования к разрабатываемому приложению.

\emph{В третьем разделе} на стадии технического проектирования представлены проектные решения для приложения.

\emph{В четвертом разделе} приводится список классов и их методов, использованных при разработке сайта, производится тестирование разработанного приложения.

В заключении излагаются основные результаты работы, полученные в ходе разработки.

В приложении А представлены фрагменты исходного кода. 
